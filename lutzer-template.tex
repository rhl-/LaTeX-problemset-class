%%%%%%%%%%%%%%%%%%%%%
% This file has been borrowed for Carl Lutzer:
% http://people.rit.edu/cvlsma/LaTeX/
%%%%%%%%%%%%%%%%%%%%%
%%%%%%%%%%%%%%%%%%%%%
%
% This file is my first attempt at providing a general 
% introductory template for LaTeX.  It's meant to provide
% some basic understanding of structure and language,
% and to demonstrate some of the basics by way of example.
% 
% I recommend compiling it so that you can see the 
% typesetting to which the comments refer.
%
% NOTE: Percent signs are used to indicate comments.
% The compiler doesn't read past a percent sign on a
% given line.
%
%%%%%%%%%%%%%%%%%%%%%

%%%%%%%%%%%%%%%%%%%%%
%
% Start by giving the compiler 
% formatting instructions
%
%%%%%%%%%%%%%%%%%%%%%

\documentclass[12pt]{report} % You may use ``article" instead of ``report"

\textheight 8 in			% Adjusts the height of the printed area
\textwidth 5 in			% Adjusts the width of the printed area
\topmargin 0 in			% Adjusts the top margin of the document
\oddsidemargin .9 in   	% Left Margin adjustment for odd pages (0 = 1" margin)
\evensidemargin .9 in  % Right Margin adjustment for even pages


\pagestyle{myheadings}				% This command says to use headers.
\markright{Template for \LaTeX~ Source Code} % This line lets you define the header
\setcounter{footnote}{0}				% The footnotes will start at 1.


%%%%%%%%%%%%%%%%%%%%%
%
% Next, we pull in some packages that define 
% particular symbols, fonts, and environments
% that are useful when typing mathematical
% documents
%
%%%%%%%%%%%%%%%%%%%%%
\usepackage{amsbsy}
\usepackage{amsthm}
\usepackage{amsfonts}
\usepackage{graphicx}
\usepackage{color}		% Allows color in text, demonstrated below.


%%%%%%%%%%%%%%%%%%%%%
%
% Next, we define some new ``environments"
% Environments are entered and exited with
% \begin{environment} and \end{environment}
% commands, as shown below in the case of 
% itemization.
%
%%%%%%%%%%%%%%%%%%%%%

\newtheorem{theorem}{THEOREM}		
\newtheorem{lemma}{LEMMA}
\newtheorem{proposition}{PROPOSITION}
\newtheorem{definition}{DEFINITION}
\newtheorem{corollary}{COROLLARY}


%%%%%%%%%%%%%%%%%%%%%
%
% Next, we define some short-hand notation
% because typing long command sequences
% over and over again is tedious, and hard on
% the fingers.
%
%%%%%%%%%%%%%%%%%%%%%


\newcommand{\mbf}{\mathbf}	% Tells the compiler to switch into math-style bold-face
\newcommand{\R}{\mathbb{R}}	% To use the symbol for the set of real numbers, type \R
\newcommand{\N}{\mathbb{N}}	% To use the symbol for the set of natural numbers type \N
\newcommand{\C}{\mathbb{C}}	% To use the symbol for the set of complex numbers type \C
\newcommand{\Z}{\mathbb{Z}}	% To use the symbol for the set of integers type \Z
\newcommand{\Q}{\mathbb{Q}}	% To use the symbol for the set of rational numbers type \Q

\newcommand{\be}{\begin{enumerate}} % Defines a short-hand notation for the tag
\newcommand{\ee}{\end{enumerate}}	% Defines a short-hand notation for the tag
\newcommand{\bi}{\begin{itemize}}		% Defines a short-hand notation for the tag
\newcommand{\ei}{\end{itemize}}		% Defines a short-hand notation for the tag


\newcommand{\ds}{\displaystyle}		% Defines a short-hand notation for the tag

\newcommand{\bx}{\hfill \rule{1.5 mm}{1.5 mm}} % Used to denote the end of something.

%%%%%%%%%%%%%%%%%%%%%
% I've included the command \limit, below, as an example of how to 
% define a new command that takes arguments.  In this case, the 
% command takes two arguments.  For example, \limit{x}{\infty} .  
% As you can see in the definition, this command gives the 
% display-style typesetting.  If you want in-line typesetting, you should 
% use \lim_{x \rightarrow \infty} instead.
%%%%%%%%%%%%%%%%%%%%%
\newcommand{\limit}[2]{{\displaystyle{\lim_{#1 \rightarrow #2}}~}}


%%%%%%%%%%%%%%%%%%%%%
% Graphics:
%
% 	If you want to include the figure "mypicture.eps" in the directory "images"
% 	I suggest that you use the command
%
% 		\includegraphics[scale=0.5,angle=-90]{images/mypicture.eps}
%
% 			or, for Mac OSX 
%
% 		\includegraphics[scale=0.5,angle=-90]{images/mypicture.pdf}
%
%%%%%%%%%%%%%%%%%%%%%

% Here are some stylistic choices
\renewcommand{\thechapter}{\arabic{chapter}}
\renewcommand{\thesection}{\thechapter}
\renewcommand{\thefootnote}{\arabic{footnote}}


%%%%%%%%%%%%%%%%%%%%%
%
% Now we begin the document.
%
%%%%%%%%%%%%%%%%%%%%%
\begin{document}
\thispagestyle{plain}	% Removes headers from this page, but puts a page number
%\thispagestyle{empty} % Removes both headers and page numbers from this page.


\begin{center}
{\Huge {\it \LaTeX~ Primer}}\\
\rule{\textwidth}{0.1mm}
\end{center}
% Because of the format I've used, the section number will reflect the chapter 
% number, which is set with the \setcounter command.  There are other 
% counters, such as enumi (which controls the first level of enumeration: 1,2,3,...) 
% and enumii (which controls the second level of enumeration 1a, 1b, 1c,...)
\setcounter{chapter}{1} 
\section{Introduction (Read Me)}
\label{intro} % This command allows me to reference the section number later.

\noindent{This} document is intended to be used as a primer.  You are welcome to 
remove the body of the document and use the headers only.  If you actually read 
this document, read with the goal of {\it learning by way of example} -- comparing 
this .pdf file with the source code that generated it.  Most everything that's done 
here is done for that reason (i.e., this isn't meant to be a narrative).  After doing 
that, you should be able to answer questions like:

\begin{itemize}
\item How can you display an equation on it's own line?  
\item How can you force a page break?  
\item How can you include color?  
\item How do you include footnotes?  
\item How do you make horizontal separators (such as the one above)?  
\item How do you makes enumerated or bulleted lists?
\item How do I typeset math symbols?
\item How do I give commands to the compiler?
\end{itemize}

I'll answer the last question right away.  Commands to the \LaTeX~ compiler  
are preceded by a backslash.  They're all over the source code (take a look).  For 
example, all of the math symbols you'll want are indicated by a command of the 
form $\backslash$name.  For example, the symbol $\phi$ is generated by typing $\backslash$phi in math mode (see Section \ref{math} for ``math mode").

As of \today, there's a nice online dictionary of symbols and ``environments" at 
\begin{center}
{\texttt http://crypto.stanford.edu/~bcwalrus/197/main.html}
\end{center}

Most of the symbols you could ever want are in the ``Math" section of the page 
(there's a link in the left frame).  However, I also recommend buying a reference 
book such as Leslie Lamport's {\it LaTeX: A Document Preparation System 
(2nd Edition)}.\\
~\\
\centerline{\shortstack{\rule{0.3\textwidth}{0.5 mm} \\ \rule{0.3\textwidth}{0.5 mm}}}

% Since I'm not using chapters, I have to increment the chapter counter manually.  
\setcounter{chapter}{2} 
\section{Paragraphs and Indentation}
\label{paragraphs}

You might notice that some of the lines in this document are indented (like this one), 
but others were not.  \LaTeX~ keeps the first line of a section left justified, but will 
otherwise  indent when it sees two carriage returns or when you're transitioning out 
of an environment (note the sentence immediately following the centered URL above).  
There are various ways to keep the compiler from indenting.  
	\begin{enumerate} % Enters the enumeration environment
	\item You can use the $\backslash$noindent command (demonstrated in the 
	source code below).  
	
	\noindent{Here} is an example of the $\backslash$noindent command in use.  
	This paragraph is aligned with the one above it, in the enumeration.  Notice that 
	the $\backslash$noindent command was placed only around the first word of 
	the paragraph.
	
	\item You can also use forced returns to end a line.\\
	~\\
	This technique was used to avoid an indentation in this line.  
	
	\item You can also avoid an indentation when exiting an ``environment" by leaving 
	no space between the end of the environment and the following paragraph.  I'll do 
	this below.
	\end{enumerate} % Exits the enumeration environment

Notice that this new paragraph is not indented.  I've put a footnote\footnote{The 
footnote command is takes one argument: your footnote.} at the bottom of the page 
in order to demonstrate its use.

\setcounter{chapter}{3}
\section{Page Breaks}

I let \LaTeX ~decide where to make the vast majority of page breaks.  It does a 
pretty good job.  Occasionally, when typing smaller documents or when I need 
better control over formatting and placement of figures, etc., I force a pagebreak 
with the command $\backslash$newpage . 

Note that there are not headers on the first page, but there are on this page.  \LaTeX~ 
omits headers on the first page but, after that, reverts to the page style defined at 
the beginning of the document (search the source code for $\backslash$pagestyle).  
The most common page styles are {\it empty}, {\it plain} and {\it myheadings}.  They 
are used to do the following:
	\bi
	\item[\it empty:] No page numbers, no headers.
	\item[\it plain:] Page numbers, but no headers.
	\item[\it myheadings:] Page numbers and headers.
	\ei

\thispagestyle{plain}
I've inserted the command $\backslash$thispagestyle$\{$plain$\}$ in the source 
code between this sentence and the list above it.  This command changes the page 
style for this one page.  After this page, the compiler will return to the default page 
style for this document.\\

The next section is about math symbols.  Notice that the section is not numbered.  
This is because I've used the command $\backslash$section* instead of 
$\backslash$section in the source code.\\
\begin{center}
\fbox{~Here, I'll use the $\backslash$newpage command.~}
\end{center}

\newpage

\setcounter{chapter}{4}
\section*{Typing Math}
\label{math}
Sections \ref{intro} and \ref{paragraphs} focused on some basic formatting, but 
you're using \LaTeX~ because you need to write a technical document involving
mathematical symbols.  This section will touch on {\it some} symbols.  Others can 
be found in any standard reference book, and many can be found at the online 
dictionary cited in Section \ref{intro}.  

\begin{enumerate} 	% Enters the enumeration environment
\setcounter{enumi}{3} 
% The enumi counter resets at zero every time you leave the enumerate 
% environment.  If you want to continue an enumeration, you have to set the 
% counter at the right starting point.

\item To move into or out of the inline math mode, use a single dollar sign \$.  
For example, the command \$$\backslash$R\$ makes the sign for the real 
numbers, $\R$.  


\item If you haven't noticed already, note that any command you want to give the 
compiler is preceded by a backslash.  Since backslashes tell the compiler that an 
instruction will follow, the backslash is a protected character.  If you want to typeset 
one, you have to use the command \$$\backslash$backslash\$

\item To move into our out of the display mode for math, use a double dollar 
sign, \$\$.  For example, the command 
\begin{center}
\$\$ $\backslash$sum\underline{~}$\{$k=1$\}\hat{~}\backslash$infty 
$\backslash$left ( $\backslash$frac$\{1\}\{2\}\backslash$right )
$\hat{~}\{\backslash$ln k$\}$, \$\$  
\end{center}
generates the display of 
$$	% Enters the display mode
\sum_{k=1}^\infty \left ( \frac{1}{2} \right )^{\ln k} ,
$$ 	% Exits the display mode
which is nicely set apart from the narrative and easy to see.

\item You might also find the following symbols useful:
$$
\int_1^{17} \frac{x^4-9x+2}{\ln x + \log_7 x +20}~dx ~~\mbox{ or }~~
{8 \choose 4} 
$$

\item The symbols
$$
\frac{d}{dx} f(x) ~~\mbox{ and }~~ f_j(x)\stackrel{w}{\rightarrow} f(x)
$$
are also useful.

\item Notice (in the source code) that the command $\backslash$mbox is used to 
enter text while in math mode, and the character $\sim$ is used to make spaces 
(you can use $\sim$ anywhere to force spaces).


\item If you want to type a matrix, use an array in the math mode.  In an array, 
you indicate the number of columns by telling the compiler how each column should 
be aligned (this is part of the argument of the array command, as demonstrated in 
the source code below).  Moving from one column to another is done with an 
ampersand.
\begin{equation} % Enters equation mode
\label{ref demo} % Allows me to reference this equation later
\left [ 
\begin{array}{cclr}
4 & -1 & -x & 0 \\
2 & 3 & -8 & -9 \\
x & i & e & \pi
\end{array}
\right ]
\end{equation} % Exits equation mode

\item Notice that the $\backslash$left and $\backslash$right commands were 
used in (\ref{ref demo}) to help delimit the matrix.  These commands, which are 
always used together, tell the compiler to enlarge the brackets to bound what's 
between them.  Without their use, the matrix would look like this:
$$
 [ 
\begin{array}{cclr}
4 & -1 & -x & 0 \\
2 & 3 & -8 & -9 \\
x & i & e & \pi
\end{array}
 ]
$$
\item Arrays can be partitioned with vertical or horizontal lines.
$$
\left [ 
\begin{array}{ccl|r}
4 & -1 & -x & 0 \\
\hline
2 & 3 & -8 & -9 \\
x & i & e & \pi \\
\end{array}
\right ]
$$

\item When dealing with matrices, the commands to produce $\cdots$, $\vdots$ 
and $\ddots$ are often useful.

\item You can suppress one side of the $\backslash$left and $\backslash$right 
pairing with a period.  For example,
$$
f(x) = \left \{ 
\begin{array}{ll}
e^{-1/x^2} & \mbox{ if } x \not = 0 \\
0 & \mbox{ if } x =0
\end{array} 
\right .
$$
Notice (in the source code) that I generated the $\{$ with the command 
$\backslash \{$.  I have to do it this way because the curly-braces are used 
to delimit the arguments of commands to the compiler, so they are also protected 
characters.

\item The tabular environment is very similar to the array environment, except 
that you use it in text mode instead of math mode.  For example,
\begin{center} % Enters the centered environment
\begin{tabular}{lr} % Enters the tabular environment
this is an example & of \\
a tabular & environment
\end{tabular} % Exits the tabular environment
\end{center} % Exits the centered environment

\item Just as with arrays, you can partition a table with vertical and horizontal lines.
\begin{center}
\begin{tabular}{|r|c|}
\hline
{\bf Item} & {\bf Cost} \\
\hline
\hline
Computer & \$ 1200 \\
\hline 
Daily intake of coffee & \$2.25 \\
\hline
\LaTeX~compiler & free \\
\hline
Hours of typing source code & priceless\\
\hline
\end{tabular}
\end{center}

\end{enumerate} % Exits the enumeration environment.  

\setcounter{chapter}{5}
\section{Referencing Equations, Sections, Etc.}
\subsection{The Benefits}

Several equations and sections have been referenced so far in this document.  It's 
silly to try to keep track of them all by hand -- that's the kind of thing we have 
computer for, and it's exactly what \LaTeX does for you.  If you're reading the 
source code along with this .pdf file, you've already seen the 
$\backslash$label$\{$name$\}$ and $\backslash$ref$\{$name$\}$ commands in 
use.  As long as you keep the same labels with the same equations, you'll never 
have to worry about keeping track of your references.  The compiler will just 
renumber them as needed.  I'll use the $\backslash$begin$\{$enumerate$\}$ 
command to demonstrate (by way of analogy).  Suppose I have the list,
\be
\item onions
\item potatoes
\item tomatoes
\item soda
\ee
but realize that I want to insert an item in the second slot.  I don't need to 
renumber the list by hand.  Instead, I just insert the new item in the appropriate 
slot and \LaTeX~ renumbers the later items for me (take a look at the source code)
\be
\item onions
\item potatoes
\item Dan Quayle
\item tomatoes
\item soda
\ee
Equations and sections, and anything else that's labeled, works the same way.  So 
instead of referencing equation (4.1) by hand (as I just did in the source code), I 
reference equation (\ref{ref demo}) by typing ($\backslash$ref$\{$ref demo$\}$).  
Now, if I were to insert a new label before equation (\ref{ref demo}), it would be 
renumbered automatically and I won't have to change all of my references to it.  
This can save a {\it lot} of time.

\subsection{Here's the Catch}

The \LaTeX compiler keeps all of the equations labels in an auxiliary file.  Every 
time you compile the document, \LaTeX generates the .aux file (or overwrites it, 
if it already exists).  

\bi
\item The first time you compile the document, the .aux file does not yet exist, 
so the compiler can't use it to cross-reference for you.  That means you have to 
compile a second time in order for the reference numbers to appear.

\item If you're working on a document that's already been compiled once or more 
times, the .aux file already exists, and \LaTeX~ will use it to generate equation 
numbers, etc.  {\it However}, the labels for anything new will not be in the .aux 
file, so the new labels won't work.  To get new labels to work, compile the document 
a second time.
\ei

As a general rule, if you want to see the basic document, compile once.  If you 
want to have all the reference numbers correct, compile a second time.  If you 
are going to print a finished copy, compile a third time (just to be sure).

\setcounter{chapter}{6}
\section{Color}
You can use basic colors like red, green, blue, and magenta without much trouble, 
but others you have to define.  I like to use colors to highlight a step that students 
might not find obvious.  For example,
\begin{eqnarray*}
{\mathcal{L}} & = & \int_0^{2\pi} \sqrt{ (1 + \sin \theta)^2 + \cos^2\theta}~d\theta \\
& = & 
\int_0^{2\pi} \sqrt{2 + 2\sin \theta}~d\theta
= 
\int_{-\pi/2}^{3\pi/2} \sqrt{2 + 2\sin \theta}~d\theta\\
& = & 
2\int_{-\pi/2}^{\pi/2} \sqrt{2 + 2\sin \theta}~d\theta
 = 
2\int_{-\pi/2}^{\pi/2} \sqrt{2 + 2\sin \theta}~{\color{blue}
\frac{\sqrt{2 - 2\sin\theta}}{\sqrt{2 - 2\sin\theta}}}~ d\theta \\
& = & 
2\int_{-\pi/2}^{\pi/2} \frac{2|\cos \theta|}{\sqrt{2 - 2\sin\theta}}~d\theta
 = 
4\int_{-\pi/2}^{\pi/2} \frac{\cos \theta}{\sqrt{2 - 2\sin\theta}}~d\theta .
\end{eqnarray*}

Notice that all of the equalities are aligned in the above string.  This is because 
I've used the eqnarray* environment.  If you want a line or two to be named, you 
should use the eqnarray environment instead, as follows:

\begin{eqnarray}
{\mathcal{L}} & = & \int_0^{2\pi} \sqrt{ (1 + \sin \theta)^2 + \cos^2\theta}~d\theta =
\int_0^{2\pi} \sqrt{2 + 2\sin \theta}~d\theta \nonumber \\
& = & 
\int_{-\pi/2}^{3\pi/2} \sqrt{2 + 2\sin \theta}~d\theta
=
2\int_{-\pi/2}^{\pi/2} \sqrt{2 + 2\sin \theta}~d\theta \nonumber \\
& = & 
2\int_{-\pi/2}^{\pi/2} \sqrt{2 + 2\sin \theta}~{\color{blue}
\frac{\sqrt{2 - 2\sin\theta}}{\sqrt{2 - 2\sin\theta}}}~ d\theta \label{demo} \\
& = & 
2\int_{-\pi/2}^{\pi/2} \frac{2|\cos \theta|}{\sqrt{2 - 2\sin\theta}}~d\theta
 = 
4\int_{-\pi/2}^{\pi/2} \frac{\cos \theta}{\sqrt{2 - 2\sin\theta}}~d\theta .\nonumber 
\end{eqnarray}
Now I can alert students to the strange factor of 1 in equation (\ref{demo}). 

\setcounter{chapter}{6}
\section{Other Resources}
There are other online tutorials for \LaTeX, some of which I'll give here.

	\begin{itemize}
	\item http://www.tug.org.in/tutorials.html
	\item http://heather.cs.ucdavis.edu/$\sim$matloff/latex.html
	\item {\scriptsize http://www.cs.cornell.edu/Info/Misc/LaTeX-Tutorial/LaTeX-Home.html}
	\item http://www.csclub.uwaterloo.ca/u/sjbmann/tutorial.html
	\end{itemize}


Each of the above websites was found by searching for ``LaTeX tutorial" at
{\texttt http://www.google.com/}.  These are good starts, but I find it helpful to have an actual book (with a good index) next to the computer while I code .tex files.


\end{document} 
